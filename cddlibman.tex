% The name of this file: cddlibman.tex
% written by by Komei Fukuda
% created March 15, 1999
%
\documentclass[11pt]{article}
\usepackage{html}
\renewcommand{\baselinestretch}{1}
\renewcommand{\arraystretch}{1}
\setlength{\oddsidemargin}{0mm}
\setlength{\textwidth}{16cm}
\setlength{\topmargin}{-15mm}
\setlength{\textheight}{235mm}
%\setlength{\headsep}{0in}
%\setlength{\headheight}{0pt}
\pagestyle{plain}

\begin{document}
\title{cddlib Reference Manual}
\author{
Komei Fukuda \\
Institute for Operations Research\\
ETH-Zentrum, CH-8092 Zurich, Switzerland\\
fukuda@ifor.math.ethz.ch \\and\\
Department of Mathematics\\
ETFL, CH-1015 Lausanne, Switzerland}
\date{ (cddlib ver. 0.80,  March 15, 1999)}

\maketitle
\begin{abstract}
This is a reference manual for cddlib-080, quickly written from the cdd-061 manual.  It is imcomplete and merely meant to be a starting point for a complete future version.  Please use the accompanying README file and test programs to complement the incompleteness.
\end{abstract}

\section{Introduction} \label{INTRODUCTION}

The program  cddlib  is an ANSI C implementation of 
the Double Description Method~\cite{mrtt-ddm-53}
for generating all vertices (i.e. extreme points)
and extreme rays of a general 
convex polyhedron given by a system of linear inequalities:
\[
   P = \{ x  \in R^{d+1}:  A  x  \ge 0, x=(x_0=1, x_1, x_2, \ldots, x_d) \}
\]
where $A$ is an $m \times d+1$ real matrix.   See, \cite{fp-ddmr-96} for
efficient implementations of the double description
method on which cddlib is based on.

cddlib is a C-library version of the previously released C-code cdd/cdd+.
In order to make this library version, a large part of the cdd source
(Version 0.61) has been rewritten.
This library version is more flexible since it can be called from any
other C programs.

One useful feature of  cddlib/cdd/cdd+ is its capability
of handling the dual (reverse)  problem without any transformation
of data.  The dual problem is known to be the 
{\em (convex) hull problem\/} which
is to obtain a linear inequality representation
of a convex polyhedron given as the Minkowski sum of 
the convex hull of a finite set of points and the nonnegative
hull of a finite set of points in $R^{d+1}$: 
$P = conv(v_1,\ldots,v_n) +  nonneg(r_1,\ldots,r_s)$, where
 the {\em Minkowski sum of two subsets $S$ and $T$} of $R^{d+1}$ is defined
as $S + T = \{ s + t \; |  s \in S \mbox{ and } t \in T \}$.
As we see in this manual, the computation can be done
in straightforward manner.  Unlike the earliear versions of
cdd/cdd+ that assume certain regularity conditions for input, cddlib.

cddlib comes with an LP code dplex to solve the general
linear programming (LP) problem to maximize (or minimize) a linear
function over polyhedron $P$.   It is useful mainly for solving 
dense LP's with large $m$ (say, up to few hundred thousands) and small $d$ 
(say, up to 100).

The program cddlib has an I/O routines that read and write files in 
{\em Polyhedra format\/} which was defined by David Avis and
the author in 1993, and has been updated in 1997.  
The program called lrs \cite{a-uglrs-97} developed by David Avis is
a C-implementation of the reverse search algorithm~\cite{af-pachv-92} 
for the same enumeration purpose, and it conforms to Polyhedra format as well.
Hopefully, this compatibility of the two programs
enables users to use both programs for the same input files
and to choose whichever is useful for their purposes.
From our experiences with relatively large problems,
the two methods are both useful and perhaps complementary
to each other.  In general, the program cdd+ tends to be
efficient for highly degenerate inputs and the program rs
tends to be efficient for nondegenerate or slightly
degenerate problems.

Although the program can be used for nondegenerate inputs,
it might not be very efficient.  For nondegenerate inputs, 
other available programs, such as the reverse search code lrs or
qhull (developed by the Geometry Center),
might be more efficient.  See Section~\ref{CODES} 
for pointers to these codes.  
The paper \cite{abs-hgach-97} contains many interesting results on polyhedral
computation and experimental results on cdd+, lrs, qhull and porta.

This program can be distributed freely under the GNU GENERAL PUBLIC LICENSE.
Please read the file COPYING carefully before using.

I will not take any responsibility of any problems you might have
with this program.  But I will be glad to receive bug reports or suggestions
at the e-mail addresses above.  Finally, if cdd+ turns out to be useful, 
please kindly inform  me of  what purposes cdd has been used for. 
I will be happy to include a list of applications in future
distribution  if I receive  enough replies.
The most powerful support for free software development
is user's appreciation and collaboration.

\section{Polyhedra H- and V-Formats (Version 1997)} \label{FORMAT}
\bigskip
Every convex polyhedron has two representations, one as
the intersection of finite halfspaces and the other
as Minkowski sum of the convex hull of finite points
and the nonnegative hull of finite directions.  These are
called H-representation and V-representation, respectively.

Naturally there are two basic Polyhedra formats, 
H-format for  H-representation and V-format for
V-representation.    These two formats are designed
to be almost indistinguishable, and in fact, one can
almost pretend one for the other.   There is some asymmetry
arising from the asymmetry of two representations.

First we start with the halfspace representation.
Let $A$ be an $m \times d$ matrix, and let $b$ be a column $m$-vector.
The Polyhedra format  ({\em  H-format} )  of 
the system $ A x \le b$ of $m$ inequalities in $d$ variables
$x =(x_1, x_2, \ldots, x_d)^T$ is

\begin{tabular}{ccl}
\\ \hline
\multicolumn{3}{l} {various comments}\\
\multicolumn{3}{l} {{\bf H-representation}}\\
\multicolumn{3}{l} {{\bf begin}}\\
 $m$ & $d+1$ & numbertype\\
 $b$ & $-A$ \\
\multicolumn{3}{l} {{\bf end}}\\
\multicolumn{3}{l} {various options} \\ \hline
\end{tabular}

\bigskip
\noindent
where numbertype can be one of integer, rational or real.
When rational type is selected, each component
of $b$ and $A$ can be specified by the usual integer expression 
or by the rational expression ``$p / q$''  or  ``$-p / q$'' where
$p$ and $q$ are arbitrary long positive integers (see the example
input file rational.ine).  In the new 1997 format,
we introduced ``H-representation'' which must appear
before ``begin''. 
There was one restriction in the old polyhedra format 
(before 1997):  the last $d$ rows must determine
a vertex of $P$.  This is obsolete now.

Now we introduce  Polyhedra  {\em V-format}.  Let $P$ be 
represented by $n$ extreme points and $s$ rays as 
$P = conv(v_1,\ldots,v_n) +  nonneg(r_1,\ldots,r_s)$.
Then the Polyhedra V-format for $P$ is defined as

\begin{tabular}{ccl}
\\ \hline
\multicolumn{3}{l} {various comments}\\
\multicolumn{3}{l} {{\bf V-representation}}\\
\multicolumn{3}{l} {{\bf begin}}\\
 $n+s$ & $d+1$ & numbertype\\
 $1$ & $v_1$  & \\
 $\vdots$ & $\vdots$  & \\
 $1$ & $v_n$  & \\
 $0$ & $r_1$  & \\
 $\vdots$ & $\vdots$  & \\
 $0$ & $r_s$  & \\
\multicolumn{3}{l} {{\bf end}}\\
\multicolumn{3}{l} {various options} \\ \hline
\end{tabular}

\bigskip
\noindent
Here we do not require that
vertices and rays are listed
separately; they can appear mixed in arbitrary
order.

When the representation statement, either ``H-representation''
or ``V-representation'', is omitted, the former
``H-representation'' is assumed.


It is strongly suggested to use the following rule for naming
H-format files and V-format files:   
\begin{description}
\item[(a)] use the filename  extension ``.ine'' for H-files (where ine stands for inequalities), and 
\item[(b)]  use the filename  extension ``.ext'' for V-files (where ext stands for extreme points/rays). 
\end{description}

In addition to the Polyhedra format, cddlib uses two options ``equality''
and ``linearity'' which allow Polyhedra files to represent equalities
and linear generators.

\section{Basic Object Types (Structures) in cddlib}  \label{DATASTR}

Here are among the most important types defined in cddlib.

\begin{verbatim}

typedef struct dd_MatrixData *dd_MatrixPtr;

typedef struct dd_MatrixData {
  dd_rowrange rowsize;
  dd_rowset linset; 
    /*  a subset of rows of linearity (ie, generators of
        linearity space for V-representation, and equations
        for H-representation. */
  dd_colrange colsize;
  dd_NumberType number;
  dd_Amatrix matrix;  
} dd_Matrix;

typedef struct dd_SetFamilyData *dd_SetFamilyPtr;

typedef struct dd_SetFamilyData {
  dd_bigrange famsize;
  dd_bigrange setsize;
  dd_SetVector set;  
} dd_SetFamily;

typedef struct dd_PolyhedraData {
  dd_RepresentationType Representation;  /* given representation */
  boolean Homogeneous;
  dd_colrange d;
  dd_rowrange m;
  dd_Amatrix A;   /* Inequality System or Generator:  m times d matrix */
  dd_NumberType Number;
  dd_ConePtr child;  /* pointing to the homogenized cone data */
  dd_rowrange m_alloc; /* allocated row size of matrix A */
  dd_colrange d_alloc; /* allocated col size of matrix A */
  dd_Arow c;           /* cost vector */

  dd_rowflag EqualityIndex;  
    /* ith component is 1 if it is equality (linearity), 0 otherwise. */

  boolean NondegAssumed;
  boolean InitBasisAtBottom;
  boolean RestrictedEnumeration;
  boolean RelaxedEnumeration;
} dd_Polyhedra;

\end{verbatim}

\section{Library Functions}  \label{LIBRARY}

The following functions of cddlib are most important. 

\begin{verbatim}
dd_MatrixPtr dd_CopyInequalities(dd_PolyhedraPtr);

dd_MatrixPtr dd_CopyGenerators(dd_PolyhedraPtr);

dd_SetFamilyPtr dd_CopyIncidence(dd_PolyhedraPtr);

dd_SetFamilyPtr dd_CopyAdjacency(dd_PolyhedraPtr);

boolean dd_DoubleDescription(dd_PolyhedraPtr);

boolean dd_DDAddInequalities(dd_PolyhedraPtr, dd_MatrixPtr);

boolean dd_PolyhedraInput(dd_ErrorType*, dd_PolyhedraPtr *);

void dd_PolyhedraLoadMatrix(dd_PolyhedraPtr *, 
  dd_RepresentationType, dd_MatrixPtr );

dp_LPPtr dp_LPInput(FILE **f, dp_ErrorType *err);  

dp_LPPtr dp_LPLoad(dp_LPConversionType,
   dp_NumberType, dp_rowrange m, dp_colrange d, dp_Amatrix, 
   dp_rowrange OBJrow, dp_colrange RHScol, dp_ErrorType *err);  
   /* 
      Load an LP safely.  This creates a copy of LP data,
      and returns a pointer to the LPDataType.  
   */

dp_LPPtr dp_LPDirectLoad(dp_LPConversionType,
   dp_NumberType, dp_rowrange m, dp_colrange d, dp_Amatrix*, 
   dp_rowrange OBJrow, dp_colrange RHScol, dp_ErrorType *err);  
   /* 
      Load an LP quickly.  This creates a direct links to A
      and a copy of other data, and returns a pointer to the LPDataType.
      This deletes the pointer A so that the loaded LPdata won't be
      affected by the user.  Use this function only when saving memory
      or time is extremely important.
   */

void dp_LPSolve(dp_LPPtr, dp_ErrorType *);

dp_LPPtr dp_MakeLPforInteriorFinding(dp_LPPtr);  

dp_LPSolutionPtr dp_LPSolutionLoad(dp_LPPtr lp);

\end{verbatim}

\section{How to Use}  \label{HOWTO}

See the examples, cdd\_test*.c and dplex\_test*.c


\section{FTP site}  \label{FTP}
An anonymous \htmladdnormallink{ftp}
{ftp://ftp.ifor.math.ethz.ch/pub/fukuda/cdd/} site for the programs is set at:
\begin{verbatim}
   ftpsite:  ftp.ifor.math.ethz.ch
   directory: pub/fukuda/cdd
   filenames: cddlib-***.tar.gz
\end{verbatim}
Since the file is compressed binary file, it is necessary to use binary mode for
file transfer.

\section{Other Useful Codes}  \label{CODES}
There are several other useful codes available for vertex enumeration and/or
convex hull computation  such as lrs, qhull, porta and irisa-polylib.
The pointers to these codes are available at
\begin{enumerate}
\item lrs by D. Avis \cite{a-uglrs-97} (C implementation of the reverse search algorithm 
\cite{af-pachv-92}). 

\item qhull by C.B. Barber \cite{bdh-qach-95} (C implementation of
the beneath-beyond method, see \cite{e-acg-87,m-cg-94},
which is the dual of the dd method). 

\item porta by T. Christof and A. L\"obel \cite{cl-porta-97} (C implementation
of the Fourier-Motzkin elimination).

\item pd by A. Marzetta \cite{m-pdcip-97} (C implementation of the primal-dual algorithm 
\cite{bfm-pdmvf-97}). 

 \item Geometry Center Software List by N. Amenta \cite{a-dcg}.

 \item Computational Geometry Pages by J. Erickson \cite{e-cgp}.

 \item Linear Programming FAQ by R. Fourer and J. Gregory \cite{fg-lpfaq-97}.

 \item ZIB Berlin polyhedral software list:\\
 \htmladdnormallink{ftp://elib.zib-berlin.de/pub/mathprog/polyth/index.html}
{ftp://elib.zib-berlin.de/pub/mathprog/polyth/index.html}.

\item Polyhedral Computation FAQ \cite{f-pcfaq-98}.
\end{enumerate}


\section*{Acknowledgements.} 
I am  grateful to Th. M. Liebling who
provided me with an ideal opportunity to visit EPFL
for the academic year 1993-1994.  Without his 
support, the present form of this program would not have existed.
There are many people who helped me to improve cdd,  in particular,
I am indebted to David Avis, Alain Prodon,  Francois Margot, Henry Crapo,
Alexander Bockmayr, David Bremner, Matthew Saltzman. 

Finally, I would like to thank both H.-J. L\"uthi (ETHZ)
 and Th. M. Liebling (EPFL) for their continuing support for 
the current new development (cdd+, cddlib).  

\bibliographystyle{alpha}

\bibliography{fukuda1,fukuda2}

\end{document}


