% The name of this file: cddlibman.tex
% written by by Komei Fukuda
% created March 15, 1999
% modified June 25, 2000
%
\documentclass[11pt]{article}
\usepackage{html}
\renewcommand{\baselinestretch}{1}
\renewcommand{\arraystretch}{1}
\setlength{\oddsidemargin}{0mm}
\setlength{\textwidth}{16cm}
\setlength{\topmargin}{-15mm}
\setlength{\textheight}{235mm}
%\setlength{\headsep}{0in}
%\setlength{\headheight}{0pt}
\pagestyle{plain}

\newcommand {\0} {{\bf 0}}

\begin{document}
\title{cddlib Reference Manual}
\author{Komei Fukuda \\
Institute for Operations Research\\
ETH-Zentrum, CH-8092 Zurich, Switzerland\\
fukuda@ifor.math.ethz.ch \\and\\
Department of Mathematics\\
ETFL, CH-1015 Lausanne, Switzerland}
\date{ (cddlib ver. 090d,  June 25, 2000)}

\maketitle
\begin{abstract}
This is a reference manual for cddlib-090c.  
It is still imcomplete but explains the most basic functions for polyhedral
representation conversion in cddlib.  
Please use the accompanying README file and test programs to complement the incompleteness.
\end{abstract}

\section{Introduction} \label{INTRODUCTION}

The program  cddlib  is an efficient implementation \cite{fp-ddmr-96}  of 
the double description Method~\cite{mrtt-ddm-53}
for generating  all vertices (i.e. extreme points)
and extreme rays of a general 
convex polyhedron given by 
a system of linear inequalities:
\[
   P = \{ x=(x_1, x_2, \ldots, x_d)^T \in R^{d}:  b - A  x  \ge 0 \}
\]
where $A$ is a given $m \times d$ real matrix and 
$b$ is a given real $m$-vector.   In the mathematical
language, the computation is the transformation
of an {\em H-representation\/} of a convex polytope
to an {\em V-representation}.  

cddlib is a C-library version of the previously released C-code cdd/cdd+.
In order to make this library version, a large part of the cdd source
(Version 0.61) has been rewritten.
This library version is more flexible since it can be called from other
C programs.

One useful feature of  cddlib/cdd/cdd+ is its capability
of handling the dual (reverse)  problem without any transformation
of data.  The dual transformation problem of a V-representation
to a minimal H-representation and is often called the 
{\em (convex) hull problem\/}.  More explicitly,
is to obtain a linear inequality representation
of a convex polyhedron given as the Minkowski sum of 
the convex hull of a finite set of points and the nonnegative
hull of a finite set of points in $R^{d+1}$: 
\[
P = conv(v_1,\ldots,v_n) +  nonneg(r_{n+1},\ldots,r_{n+s}), 
\]
where
 the {\em Minkowski sum of two subsets $S$ and $T$} of $R^{d+1}$ is defined
as 
\[
S + T = \{ s + t \; |  s \in S \mbox{ and } t \in T \}.
\]
As we see in this manual, the computation can be done
in straightforward manner.  Unlike the earlier versions of
cdd/cdd+ that assume certain regularity conditions for input, 
cddlib is designed to do a correction transformation for any general input.
The user must be aware of the fact that in certain cases the
transformation is not unique and there are polyhedra with
infinitely many representations.  For example, a line
segment (1-dimensional polytope) in $R^3$ has infinitely
many minimal H-representations, and a halfspace in the same space
has infinitely many minimal V-representations.  cddlib generates
merely one minimal representation.

cddlib comes with an LP code to solve the general
linear programming (LP) problem to maximize (or minimize) a linear
function over polyhedron $P$.   It is useful mainly for solving 
dense LP's with large $m$ (say, up to few hundred thousands) and small $d$ 
(say, up to 100).  It implements a revised dual simplex method that
updates $(d+1)\times (d+1)$ matrix for a pivot operation.

The program cddlib has an I/O routines that read and write files in 
{\em Polyhedra format\/} which was defined by David Avis and
the author in 1993, and has been updated in 1997.  
The program called lrs \cite{a-uglrs-97} developed by David Avis is
a C-implementation of the reverse search algorithm~\cite{af-pachv-92} 
for the same enumeration purpose, and it conforms to Polyhedra format as well.
Hopefully, this compatibility of the two programs
enables users to use both programs for the same input files
and to choose whichever is useful for their purposes.
From our experiences with relatively large problems,
the two methods are both useful and perhaps complementary
to each other.  In general, the program cdd+ tends to be
efficient for highly degenerate inputs and the program rs
tends to be efficient for nondegenerate or slightly
degenerate problems.

Although the program can be used for nondegenerate inputs,
it might not be very efficient.  For nondegenerate inputs, 
other available programs, such as the reverse search code lrs or
qhull (developed by the Geometry Center),
might be more efficient.  See Section~\ref{CODES} 
for pointers to these codes.  
The paper \cite{abs-hgach-97} contains many interesting results on polyhedral
computation and experimental results on cdd+, lrs, qhull and porta.

This program can be distributed freely under the GNU GENERAL PUBLIC LICENSE.
Please read the file COPYING carefully before using.

I will not take any responsibility of any problems you might have
with this program.  But I will be glad to receive bug reports or suggestions
at the e-mail addresses above.  Finally, if cdd+ turns out to be useful, 
please kindly inform  me of  what purposes cdd has been used for. 
I will be happy to include a list of applications in future
distribution  if I receive  enough replies.
The most powerful support for free software development
is user's appreciation and collaboration.

\section{Polyhedra H- and V-Formats (Version 1999)} \label{FORMAT}
\bigskip
Every convex polyhedron has two representations, one as
the intersection of finite halfspaces and the other
as Minkowski sum of the convex hull of finite points
and the nonnegative hull of finite directions.  These are
called H-representation and V-representation, respectively.

Naturally there are two basic Polyhedra formats, 
H-format for  H-representation and V-format for
V-representation.    These two formats are designed
to be almost indistinguishable, and in fact, one can
almost pretend one for the other.   There is some asymmetry
arising from the asymmetry of two representations.

First we start with the H-representation.
Let $A$ be an $m \times d$ matrix, and let $b$ be a column $m$-vector.
The Polyhedra format  ({\em  H-format} )  of 
the system  $\; b - A x \ge \0\;$ of $m$ inequalities in $d$ variables
$x =(x_1, x_2, \ldots, x_d)^T$ is

\begin{tabular}{ccl}
\\ \hline
\multicolumn{3}{l} {various comments}\\
\multicolumn{3}{l} {{\bf H-representation}}\\
\multicolumn{3}{l} {{\bf (linearity $t\;$ $i_1\;$  $i_2\;$  $\ldots$ $\;i_t$)}}\\
\multicolumn{3}{l} {{\bf begin}}\\
 $m$ & $d+1$ & numbertype\\
 $b$ & $-A$ \\
\multicolumn{3}{l} {{\bf end}}\\
\multicolumn{3}{l} {various options} \\ \hline
\end{tabular}

\bigskip
\noindent
where numbertype can be one of integer, rational or real.
When rational type is selected, each component
of $b$ and $A$ can be specified by the usual integer expression 
or by the rational expression ``$p / q$''  or  ``$-p / q$'' where
$p$ and $q$ are arbitrary long positive integers (see the example
input file rational.ine).  In the 1997 format,
we introduced ``H-representation'' which must appear
before ``begin''. 
There was one restriction in the old polyhedra format 
(before 1997):  the last $d$ rows must determine
a vertex of $P$.  This is obsolete now.

In the new 1999 format, we added the possibility of specifying {\bf linearity\/}.
This means that
for H-representation, some of the input rows can be specified as  {\bf equalities}:  
$b_{i_j} - A_{i_j} = 0 \;$ for all $j=1,2, \ldots, t$.
The linearity line may be omitted if there are no equalities.

Option lines can be used to control computation of a specific program.
In particular both cdd and lrs use the option lines to represent
a linear objective function.  See the attached LP files, samplelp*.ine.

\bigskip
Next we define Polyhedra  {\em V-format}.  Let $P$ be 
represented by $n$ gerating points and $s$ generating directions (rays) as 
$P = conv(v_1,\ldots,v_n) +  nonneg(r_{n+1},\ldots,r_{n+s})$.
Then the Polyhedra V-format for $P$ is 

\begin{tabular}{cll}
\\ \hline
\multicolumn{3}{l} {various comments}\\
\multicolumn{3}{l} {{\bf V-representation}}\\
\multicolumn{3}{l} {({\bf linearity $t\;$ $i_1\;$  $i_2\;$  $\ldots$ $\;i_t$ })}\\
\multicolumn{3}{l} {{\bf begin}}\\
 $n+s$ & $d+1$ & numbertype\\
 $1$ & $v_1$  & \\
 $\vdots$ & $\vdots$  & \\
 $1$ & $v_n$  & \\
 $0$ & $r_{n+1}$  & \\
 $\vdots$ & $\vdots$  & \\
 $0$ & $r_{n+s}$  & \\
\multicolumn{3}{l} {{\bf end}}\\
\multicolumn{3}{l} {various options} \\ \hline
\end{tabular}

\bigskip
\noindent
Here we do not require that
vertices and rays are listed
separately; they can appear mixed in arbitrary
order.

Linearity for V-representation specifies a subset of generators
whose coefficients are relaxed
to be {\bf free}:  for all $j=1,2, \ldots, t$, the $k=i_j$th generator ($v_{k}$ or $r_k$ whichever is the $i_j$th generator) is a free generator. 
This means for each such a ray $r_k$, 
the line generated by $r_k$ is in the polyhedron,
and for each such a vertex $v_k$, its coefficient is no longer nonnegative
but still the coefficients for all $v_i$'s must sum up to one. 

When the representation statement, either ``H-representation''
or ``V-representation'', is omitted, the former
``H-representation'' is assumed.

It is strongly suggested to use the following rule for naming
H-format files and V-format files:   
\begin{description}
\item[(a)] use the filename  extension ``.ine'' for H-files (where ine stands for inequalities), and 
\item[(b)]  use the filename  extension ``.ext'' for V-files (where ext stands for extreme points/rays). 
\end{description}


\section{Basic Object Types (Structures) in cddlib}  \label{DATASTR}

Here are the types (defined in cddtypes.h) that are 
important for the cddlib user.  The most important one, dd\_MatrixType,
is to store a Polyhedra data in a straightforward manner.
Once the user sets up a (pointer to) dd\_MatrixType data,
he/she can load the data to an internal data type (dd\_PolyhedraType)
by using functions described in the next section, and apply
the double descrition method to get another representation.
As an option dd\_MatrixType can save a linear objective function
to be used by a linear programming solver.

\begin{verbatim}

typedef long dd_rowrange;
typedef long dd_colrange;
typedef long dd_bigrange;

typedef set_type dd_rowset;   /* set_type defined in setoper.h */
typedef set_type dd_colset;
typedef long *dd_rowindex;   
typedef int *dd_rowflag;   
typedef long *dd_colindex;
typedef mytype **dd_Amatrix;  /* mytype is either GMP mpq_t or 1-dim double array. */
typedef mytype *dd_Arow;

typedef enum {
  Real, Rational, Integer, Unknown
} dd_NumberType;

typedef enum {
  Inequality, Generator, Unspecified
} dd_RepresentationType;

typedef enum {
  InProgress, AllFound, RegionEmpty
} dd_CompStatusType;

typedef enum {
  DimensionTooLarge, ImproperInputFormat, 
  NegativeMatrixSize, EmptyVrepresentation,
  IFileNotFound, OFileNotOpen, NoLPObjective, NoRealNumberSupport, NoError
} dd_ErrorType;

typedef enum {
  LPnone=0, LPmax, LPmin
} dd_LPObjectiveType;

typedef enum {
  LPSundecided, Optimal, Inconsistent, DualInconsistent,
  StrucInconsistent, StrucDualInconsistent,
  Unbounded, DualUnbounded
} dd_LPStatusType;

typedef struct matrixdata *dd_MatrixPtr;
typedef struct matrixdata {
  dd_rowrange rowsize;
  dd_rowset linset; 
    /*  a subset of rows of linearity (ie, generators of
        linearity space for V-representation, and equations
        for H-representation. */
  dd_colrange colsize;
  dd_RepresentationType representation;
  dd_NumberType numbtype;
  dd_Amatrix matrix;
  dd_LPObjectiveType objective;
  dd_Arow rowvec;
}  dd_MatrixType;

typedef struct setfamily *dd_SetFamilyPtr;
typedef struct setfamily {
  dd_bigrange famsize;
  dd_bigrange setsize;
  dd_SetVector set;  
} dd_SetFamilyType;

typedef struct lpsolution *dd_LPSolutionPtr;
typedef struct lpsolution {
  dd_DataFileType filename;
  dd_LPObjectiveType objective;
  dd_LPSolverType solver; 
  dd_rowrange m;
  dd_colrange d;
  dd_NumberType numbtype;

  dd_LPStatusType LPS;  /* the current solution status */
  mytype optvalue;  /* optimal value */
  dd_Arow sol;   /* primal solution */
  dd_Arow dsol;  /* dual solution */
  dd_colindex nbindex;  /* current basis represented by nonbasic indices */
  dd_rowrange re;  /* row index as a certificate in the case of inconsistency */
  dd_colrange se;  /* col index as a certificate in the case of dual inconsistency */
  long pivots[4]; 
   /* pivots[0]=setup (to find a basis), pivots[1]=PhaseI or Criss-Cross,
      pivots[2]=Phase II, pivots[3]=Anticycling */
  long total_pivots;
} dd_LPSolutionType;

\end{verbatim}

\section{Library Functions}  \label{LIBRARY}

Here we list some of the most important library functions/procedures. 
In the following table, {\tt poly} is of type {\tt dd\_PolyhedraPtr},
{\tt matrix}, {\tt matrix1} and {\tt matrix2} are of type {\tt dd\_MatrixPtr},
{\tt err} is of type {\tt dd\_ErrorType*}, {\tt ifile} and {\tt ofile}
are of type {\tt char*}, {\tt A} is of type {\tt dd\_Amatrix},
{\tt d} is of type {\tt dd\_colrange}, {\tt m} is of type {\tt dd\_rowrange},
{\tt x} is of type {\tt mytype}, {\tt a} is of type {\tt signed long integer},
{\tt b} is of type {\tt double},
{\tt setfam} is of type {\tt dd\_SetFamilyPtr},
{\tt solver} is of type {\tt dd\_LPSolverType}.

\subsection{Library Initialization}  \label{Initialization}

\begin{description}

\item[{\tt void dd\_set\_global\_constants(void)}]:\\
This is to set the global constants such as {\tt dd\_zero},
{\tt dd\_purezero} and
{\tt dd\_one} for sign recognition and basic arithmetic
operations.  {Every program to use cddlib must call this function}
before doing any computation.    Just call this once.
\end{description}

\subsection{Core Functions}  \label{CoreLibrary}

There are two types of core functions in cddlib.  The first type
runs the double description (DD) algorithm and does a representation
conversion of a specified polyhedron.  The standard header
for this type is {\tt dd\_DD*}.  The second type solves
an linear program and the standard naming is
{\tt dd\_LP*}.   Both computations are nontrivial
and the users (especially for the DD algorithm) must
know that there is a serous limit in the sizes of problems
that can be practically solved. 
Please check *.ext and *.ine files that come with cddlib to get
ideas of tractable problems. 


\begin{description}
\item[{\tt dd\_PolyhedraPtr dd\_DDMatrix2Poly(matrix, err)}]:\\
Store the representation given by {\tt matrix} in a polyhedra data, and
generate the second representation of {\tt *poly}.  It returns
a pointer to the data. {\tt *err}
returns {\tt dd\_NoError} if the computation terminates normally.  Otherwise,
it returns a value according to the error occured.

\item[{\tt dd\_DDInputAppend(\&poly, matrix, \&err)}]:\\
Modify the input representation in {\tt *poly}
by appending the matrix of {\tt *matrix}, and compute
the second representation.  The number of columns in
{\tt *matrix} must be equal to the input representation.

\item[{\tt boolean dd\_LPSolve(lp, solver, err)}]:\\
Solve {\tt lp} by the algorithm {\tt solver} and save
the solututions in {\tt *lp}.  Unlike the earlier versions
(dplex, cdd+), it can deal with equations and totally zero right
hand sides.
\end{description}


\subsection{Data Manipulations}  \label{DataLibrary}

\begin{description}

\item[{\tt void dd\_set\_si(x, a)}]:\\
This is to set a {\tt mytype} variable {\tt x} to the
value of signed long integer {\tt a}.  One
 cannot use such expressions as {\tt x=(mytype)a}.
This is a price to pay for the generality:
cddlib uses an abstract number type ({\tt mytype}) 
so that it can compute with various 
number types such as C double and GMP rational.
User can easily add a new number type by redefining
arithmetic operations in cddmp.h and cddmp.c.

\item[{\tt void dd\_set\_d(x, b)}]:\\
This is to set a {\tt mytype} variable {\tt x} to the
value of double {\tt b}.  This is available only
when the library is compiled without {\tt -DGMPRATIONAL}
compiler option.

\item[{\tt dd\_MatrixPtr dd\_PolyFile2Matrix (f, err)}]:\\
Read a Polyhedra data from {\tt f} and store it in {\tt matrixdata}
and return a pointer to the data.

\item[{\tt dd\_MatrixPtr dd\_CopyInequalities(poly)}]:\\
Copy the inequality representation pointed by poly to {\tt matrixdata}
and return {\tt dd\_MatrixPtr}.

\item[{\tt dd\_MatrixPtr dd\_CopyGenerators(poly)}]:\\ 
Copy the generator representation pointed by poly to {\tt matrixdata}
and return {\tt dd\_MatrixPtr}.

\item[{\tt dd\_SetFamilyPtr dd\_CopyIncidence(poly)}]:\\ 
Copy the incidence representation of the computed representation
pointed by poly to {\tt setfamily}
and return {\tt dd\_SetFamilyPtr}.  The computed representation is
{\tt Inequality} if the input is {\tt Generator}, and the vice visa.

\item[{\tt dd\_SetFamilyPtr dd\_CopyAdjacency(poly)}]:\\ 
Copy the adjacency representation of the computed representation
pointed by poly to {\tt setfamily}
and return {\tt dd\_SetFamilyPtr}.  The computed representation is
{\tt Inequality} if the input is {\tt Generator}, and the vice visa.

\item[{\tt dd\_SetFamilyPtr dd\_CopyInputIncidence(poly)}]:\\ 
Copy the incidence representation of the input representation
pointed by poly to {\tt setfamily}
and return {\tt d\_SetFamilyPtr}.

\item[{\tt dd\_SetFamilyPtr dd\_CopyInputAdjacency(poly)}]:\\ 
Copy the adjacency representation of the input representation
pointed by poly to {\tt setfamily}
and return {\tt d\_SetFamilyPtr}.

\item[{\tt dd\_LPPtr dd\_MakeLPforInteriorFinding(dd\_LPPtr)}]:\\
Set up an LP to find an interior point of the feasible region of {\tt lp}
and return a pointer to the LP.

\item[{\tt dd\_LPPtr dd\_Matrix2LP(matrix, err)}]:\\
Load {\tt matrix} to {\tt lpdata} and return a pointer to the data.

\item[{\tt dd\_LPSolutionPtr dd\_CopyLPSolution(lp)}]:\\
Load the solutions of {\tt lp} to {\tt lpsolution} and
return a pointer to the data.  This replaces the old name
{\tt dd\_LPSolutionLoad(lp)}.

\item[{\tt dd\_MatrixPtr dd\_CopyMatrix(matrix)}]:\\
Make a copy of matrixdata pointed by {\tt matrix} and return
a pointer to the copy.

\item[{\tt dd\_MatrixPtr dd\_AppendMatrix(matrix1, matrix2)}]:\\
Make a matrixdata by copying {\tt *matrix1} and appending
the matrix in {\tt *matrix2} and return
a pointer to the data.  The colsize must be equal in
the two input matrices.  It returns a {\tt NULL} pointer
if the input matrices are not appropriate.
Its {\tt rowsize} is set to
the sum of the rowsizes of {\tt matrix1} and {\tt matrix2}.
 The new matrixdata inherits everything else
(i.e. numbertype, representation, etc)
from the first matrix. 

\item[{\tt void dd\_FreePolyhedra(poly)}]:\\
Free memory allocated to {\tt poly}.

\item[{\tt void dd\_FreeLPData(lp)}]:\\
Free memory allocated to {\tt lp}.


\end{description}

\subsection{Input/Output Functions}  \label{IOLibrary}
\begin{description}

\item[{\tt boolean dd\_DDFile2File(ifile, ofile, err)}]:\\
Compute the representation conversion for a polyhedron given
by a Polyhedra file ifile, and write the other representation
in a Polyhedra file ofile.  {\tt *err}
returns {\tt dd\_NoError} if the computation terminates normally.  Otherwise,
it returns a value according to the error occured.

\item[{\tt void dd\_WriteMatrix(f, matrix)}]:\\
Write {\tt  matrix} to stream {\tt f}.

\item[{\tt void dd\_WriteNumber(f, x)}]:\\
Write {\tt x} to stream {\tt f}.  If {\tt x} is of GMP mpq\_t rational $p/q$,
the output is $p/q$.  If it is of C double, it is formated as a double float
with a decimal point.

\item[{\tt void dd\_WritePolyFile(f, poly)}]:\\
Write {tt poly} to stream {\tt f} in Polyhedra format.

\item[{\tt void dd\_WriteErrorMessages(f, err)}]:\\
Write error messages given by {\tt err} to stream {\tt f}.

\item[{\tt void dd\_WriteSetFamily(f, setfam)}]:\\
Write the set family pointed by {\tt setfam} to stream {\tt f}.
For each set, it outputs its index, its cardinality,
a colon ``:'' and a ordered list of its elements.

\item[{\tt void dd\_WriteSetFamilyCompressed(f, setfam)}]:\\
Write the set family pointed by {\tt setfam} to stream {\tt f}.
For each set, it outputs its index, its cardinality or the
negative of the cardinality, a colon ``:''
 and the elements in the set or its complements whichever is smaller.
Whenever it outputs the complements, the cardinality is negated
so that there is no ambiguity.
This will be considered standard for
outputing incidence (*.icd, *ecd) and adjacency 
(*.iad, *.ead) data in cddlib.   But there is some minor incompatibility
with cdd/cdd+ standalone codes.

\item[{\tt void dd\_WriteProgramDescription(f)}]:\\
Write the cddlib version information to stream {\tt f}.

\item[{\tt void dd\_WriteDDTimes(f, poly)}]:\\
Write the representation conversion time information on {\tt poly}
 to stream {\tt f}.

\end{description}

\subsection{Obsolete Functions}  \label{ObsoleteFunctions}
\begin{description}
\item[{\tt boolean dd\_DoubleDescription(poly, err)}]: 
(removed in Version 0.90c)\\
The new function
{\tt dd\_DDMatrix2Poly(matrix, err)} (see Section~\ref{CoreLibrary}) 
replaces (and actually combines) both this and 
{\tt dd\_Matrix2Poly(matrix, err)}.

\item[{\tt dd\_PolyhedraPtr dd\_Matrix2Poly(matrix, err)}]: 
(removed in Version 0.90c)\\
See above for the reason for removal.

\item[{\tt dd\_LPSolutionPtr dd\_LPSolutionLoad(lp)}]:
(renamed in Version 0.90c)\\
This function is now called {\tt dd\_CopyLPSolution(lp)}.

\end{description}

\section{How to Use}  \label{HOWTO}

See the examples, testcdd*.c , testlp*.c and simplecdd.c.


\section{FTP site}  \label{FTP}
An anonymous \htmladdnormallink{ftp}
{ftp://ftp.ifor.math.ethz.ch/pub/fukuda/cdd/} site for the programs is set at:
\begin{verbatim}
   ftpsite:  ftp.ifor.math.ethz.ch
   directory: pub/fukuda/cdd
   filenames: cddlib-***.tar.gz
\end{verbatim}
Since the file is compressed binary file, it is necessary to use binary mode for
file transfer.

\section{Other Useful Codes}  \label{CODES}
There are several other useful codes available for vertex enumeration and/or
convex hull computation  such as lrs, qhull, porta and irisa-polylib.
The pointers to these codes are available at
\begin{enumerate}
\item lrs by D. Avis \cite{a-uglrs-97} (C implementation of the reverse search algorithm 
\cite{af-pachv-92}). 

\item qhull by C.B. Barber \cite{bdh-qach-95} (C implementation of
the beneath-beyond method, see \cite{e-acg-87,m-cg-94},
which is the dual of the dd method). 

\item porta by T. Christof and A. L{\"o}bel \cite{cl-porta-97} (C implementation
of the Fourier-Motzkin elimination).

\item IRISA polyhedral library by D.K. Wilde
\cite{w-ldpo-93b} (C implementation
of a variation of the dd algorithm).

\item pd by A. Marzetta \cite{m-pdcip-97} (C implementation of the primal-dual algorithm 
\cite{bfm-pdmvf-97}). 

 \item Geometry Center Software List by N. Amenta \cite{a-dcg}.

 \item Computational Geometry Pages by J. Erickson \cite{e-cgp}.

 \item Linear Programming FAQ by R. Fourer and J. Gregory \cite{fg-lpfaq-97}.

 \item ZIB Berlin polyhedral software list:\\
 \htmladdnormallink{ftp://elib.zib-berlin.de/pub/mathprog/polyth/index.html}
{ftp://elib.zib-berlin.de/pub/mathprog/polyth/index.html}.

\item Polyhedral Computation FAQ \cite{f-pcfaq-98}.
\end{enumerate}


\section*{Acknowledgements.} 
I am  grateful to Th. M. Liebling who
provided me with an ideal opportunity to visit EPFL
for the academic year 1993-1994.  Without his 
support, the present form of this program would not have existed.
There are many people who helped me to improve cdd, in particular,
I am indebted to David Avis, 
Alexander Bockmayr, David Bremner, 
Henry Crapo, Istvan Csabai, 
Francois Margot,
Alain Prodon, 
Shawn Rusaw, Matthew Saltzman 
and Masanori Sato. 

Finally, I would like to thank both H.-J. L\"uthi (ETHZ)
 and Th. M. Liebling (EPFL) for their continuing support for 
the current new developments (cdd+, cddlib).  

\bibliographystyle{plain}

\bibliography{fukuda1,fukuda2}

\end{document}


